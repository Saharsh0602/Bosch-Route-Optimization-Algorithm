\documentclass[ % the name of the author
                    author={Callum Mann},
                % the degree programme
                % the dissertation    title (which cannot be blank)
                     title={Genetic algorithm for the CVRP},
                % the dissertation subtitle (which can    be blank)
                  subtitle={Capacitated Vehicle Routing Problem},
                % the dissertation     type
                      type={Heuristic},
                % the year of submission
                      year={2016}]{poster}

\usepackage[]{algorithm2e}

\begin{document}

% -----------------------------------------------------------------------------

\begin{frame}{}

% \begin{columns}[t]
%   \begin{column}{0.900\linewidth}
%   \begin{block}{\Large Capacitated Vehicle Routing Problem}
%     \vspace{10cm}
%   \end{block}
%   \end{column}
% \end{columns}

\begin{columns}[t]
  \begin{column}{0.422\linewidth}
    \begin{block}{\Large Representation}
      The representation of solutions (\textit{chromosomes}) are lists of tours that visit
      customers. The tour may exceed the trucks capacity making the solution infeasible,
      however there is a penalty term associated with each solution that affects their fitness.
      The benefit of this representation is that it allows genetic operators to arrange
      routes more freely and potentially generate much more optimal offsprings, as
      the optimal solutions are often very close to being infeasible.


    \end{block}
    \vspace{1cm}

    \begin{block}{\Large Initial Population}
      The initial population is designed to contain optimised routes with some randomness.
      The purpose of some randomness in the initial solution is so that the population does not
      converge too quickly resulting in cost stagnation. \vspace{1cm} \\
      The population is created by choosing a random starting node, and performing nearest neighbour
      until the capacity of the truck is reached. With some probability, the nearest neighbour will favour
      routes closer to the depot, so that the population is not identical at the beginning. After these routes have been
      created, they are passed into the \textbf{2-Opt}, to optimise the order of the routes.
      These routes then have a starting cost of around 6500, depending on the randomness parameter.
    \end{block}
    \vspace{1cm}

    \vfill

    \begin{block}{\Large Generation Step}
      The population is changed every step, by removing an undesirable solution and
      replacing it with a new child that ideally has a better fitness value. The population
      is stored in a priority queue and is advanced by the following steps:

      \SetKwProg{Fn}{def}{}{end}
      \SetKwFunction{Fgst}{GenerationStep}
      \SetKwFunction{FBOC}{BiggestOverlap}
      \SetKwFunction{FTwoOpt}{2-Opt}
      \SetKwFunction{FREP}{Repair}
      \SetKwFunction{FINT}{Introduce}

      \begin{minipage}{20cm}
        \begin{algorithm}[H]
          \LinesNumbered
          \Fn(){\Fgst{}}{
            \For{p1 in best $n$}{
              \For{p2 in best $n$}{
                $child \longleftarrow \FBOC{p1, p2}$ \\
                \For{some iterations}{
                  $child \longleftarrow \FBOC{p1, child}$
                }
                \FTwoOpt{child} \\
                \FREP{child} \\
                \FINT{child}
              }
            }
          }
        \end{algorithm}
      \end{minipage}
      \vspace{1cm}

      Both P1 and P2 typically have high fitness.
      This is done to maximise the chance of creating new optimal solutions. The children
      are created from a crossover in P1 and P2 that may be repeated several times between
      P1 and themselves. This increases the chance of improved children per iteration.
      \vspace{1cm} \\
      The children also passed through \textbf{2-Opt} after the operators as
      they often leave some room for local optimisation after routes have been
      moved around. The children are partially repaired at the end of the generation
      so that routes cannot become unsightly,
      although typically only feasible solutions are optimal in later generations
      due to the penalty term outweighing reduced cost.

    \end{block}
    \vspace{1cm}

  \end{column}

  \begin{column}{0.450\linewidth}
    \begin{block}{\Large Local Search: 2-Opt $^{[2]}$}
      The order in which a truck will visit customers may be very unoptimised, so the \textbf{2-Opt} is used
      to search locally \textit{within} the solution for lower costs. The intuition behind the algorithm
      is simple; it searches for places in the tours where one path crosses over another, and removes this cross.
      An implementation of this found in other literature$^{[0]}$ called this a method of steepest improvement.
    \end{block}
    \vspace{1cm}

    \vfill

    \begin{block}{\Large Crossover: Biggest Route Overlap $^{[0]}$}
      In this crossover a random subroute of one solution is placed into another
      solution. First, the bounding boxes for all tours are calculated, then
      the bounding boxes of the tours in the destination solutions are compared
      with the random subroute. Out of the routes with maximal bounding box overlap,
      the route with minimal demand is chosen for the subroute to be placed into optimally. \\ \vspace{1cm}
      It was found that this crossover is far superior to crossovers such as \textbf{pmx},
      which does not tailor the crossover mechanic to the TSP problem. Methods based purely
      on randomness perform poorly as the solution becomes more optimal.
    \end{block}
    \vspace{1cm}

    \begin{block}{\Large Mutation: Minimal cost insertion}
      This mutation selects a random customer from the solution and places it into
      another route. The selected customer is inserted between the two customers
      that minimise distance travelled. This typically helps later in the
      algorithm when moving whole subroutes causes too much cost increase.
    \end{block}
    \vspace{1cm}

    \begin{block}{\Large Evaluation}
      The lowest cost ever achieved by the algorithm was $5889$ for the \textit{fruitybun250}
      dataset, which represents a solution within $5.5\%$ of the best known value ($5583$). As with many GAs,
      the downfall is population convergence and stagnation, and other heuristics such as TABUROUTE$^{[1]}$
      can achieve $1\%$ within the optimal. \\ \vspace{1cm}
      An improvement to the algorithm would be found
      in the crossover operator, more specifically one may be able to track the history
      of successful operations and use this in future generations. This decreases the reliance
      on randomness and would target tours that are not optimal based on previous
      evidence.
    \end{block}

    \begin{block}{\Large References}
      [0] - A. S. Bjarnadottir (2004), \textit{Solving the Vehicle Routing Problem with Genetic Algorithms},  \\
      \vspace{0.5cm}
      $[1]$ - M. Gendreau et. al (1994), \textit{A Tabu Search Heuristic For The Vehicle Routing Problem},\\
      $[2]$ - G. A. CROES (1958). \textit{A method for solving traveling salesman problems} \\
    \end{block}
    \vspace{1cm}
  \end{column}
\end{columns}

\vfill

\end{frame}

% -----------------------------------------------------------------------------

\end{document}
